% \iffalse meta-comment
%
% File: onecell.dtx Copyright (C) 2017-2018 Jonathan P. Spratte
%
% It may be distributed and/or modified under the conditions of the LaTeX
% Project Public License (LPPL), either version 1.3c of this license or (at your
% option) any later version.  The latest version of this license is in the file
%
%   https://www.latex-project.org/lppl.txt
%
% ------------------------------------------------------------------------------
%
%<*driver>
\def\nameofplainTeX{plain}
\ifx\fmtname\nameofplainTeX\else
  \expandafter\begingroup
\fi
\input l3docstrip.tex
\askforoverwritefalse
\preamble

--------------------------------------------------------------
onecell -- place contents into a single cell
E-mail: jspratte@yahoo.de
Released under the LaTeX Project Public License v1.3c or later
See http://www.latex-project.org/lppl.txt
--------------------------------------------------------------

Copyright (C) 2017-2018 Jonathan P. Spratte

This  work may be  distributed and/or  modified under  the conditions  of the
LaTeX Project Public License (LPPL),  either version 1.3c  of this license or
(at your option) any later version.  The latest version of this license is in
the file:

  http://www.latex-project.org/lppl.txt

This work is "maintained" (as per LPPL maintenance status) by
  Jonathan P. Spratte.

This work consists of the file  onecell.dtx
and the derived files           onecell.ins,
                                onecell.pdf and
                                onecell.sty.

\endpreamble
% stop docstrip adding \endinput
\postamble
\endpostamble
\generate{\file{onecell.sty}{\from{onecell.dtx}{pkg}}}
\ifx\fmtname\nameofplainTeX
  \expandafter\endbatchfile
\else
  \expandafter\endgroup
\fi
%</driver>
%
%<*driver|pkg>
\RequirePackage{xparse,array,l3keys2e}
%</driver|pkg>
%
%<*driver>
\ProvidesFile{onecell.dtx}[2018/02/04 Place contents into a single cell]
\documentclass{l3doc}
\usepackage{enumitem}
\newenvironment{options}
  {\begin{description}[style=nextline,font=\normalfont\ttfamily]}
  {\end{description}}
\begin{document}
  \DocInput{onecell.dtx}
\end{document}
%</driver>
%
%<*pkg>
\ProvidesExplPackage
  {onecell} {2017/09/18}
  {v0.1}    {Place contents into a single cell}
%</pkg>
% \fi
%
% \title{The \pkg{onecell} package}
% \author{Jonathan P. Spratte\thanks{E-mail: jspratte@yahoo.de}}
% \date{Released 2018/02/04}
% \maketitle
% \tableofcontents
%
% \begin{documentation}
%
% \section{Introduction}
%
% This is a package to put many cells into a single one inside of a
% \env{tabular} environment. Its purpose is similar to that of \pkg{makecell}
% but should be more versatile in using different column specifiers and more
% than one column. It is in no way meant to supersede \pkg{makecell}.
%
% It is written as a \enquote{self-contained} docstrip file: executing
% |latex onecell.dtx| generates the \file{onecell.sty} file and typesets this
% documentation; execute |tex onecell.dtx| to only generate \file{onecell.sty}.
%
% \section{Documentation}
%
% \subsection{Column Specifier}
%
% \begin{function}{\onecellCycle}
%   This column specifier cycles the column specifiers behind it until every
%   used column is defined. It must be preceded by at least one valid column
%   specifier. If it is already defined when the package tries to define it an
%   error message is displayed. The definition is omitted and the column
%   specifier doesn't get used automatically.
% \end{function}
%
% \subsection{Macros}
%
% \begin{function}{\onecell}
%   \begin{syntax}
%     \cs{onecell}\meta{*}\meta{!}\oarg{columns}\oarg{alignment}\marg{contents}
%   \end{syntax}
%   This command is the actual macro typesetting \meta{contents} as one cell.
%   For this a \env{tabular} environment is used. By default you can use any
%   number of rows and columns in \meta{contents} without taking care. This is
%   done using |\onecellCycle|. As a side effect, this would also cycle over
%   |@{}| at the end of the column specifications. Because of this, by default
%   there is no such code to prevent the leading and trailing |\tabcolsep| but a
%   |\hspace*{-\tabcolsep}| before and after the \env{tabular}.
%
%   If |*| is given those |\hspace*{-\tabcolsep}|s are not used.
%   If you specify \meta{columns} what in fact is used is
%   \texttt{\{\meta{columns}\cs{onecellCycle}\meta{columns}\}}. The default is
%   also used if you give an empty argument. The package default is |c|.
%   Using |!| results in the literal interpretation of \meta{columns} so without
%   adding |\onecellCycle| to it.
%   Finally the \meta{alignment} is the vertical alignment which is directly
%   given to \env{tabular}.
% \end{function}
%
% \begin{function}{\onecellOptions}
%   \begin{syntax}
%     \cs{onecellOptions}\marg{options}
%   \end{syntax}
%   Lets you change the default behaviour of \cs{onecell}. You can specify any
%   options listed in \autoref{sec:options}.
% \end{function}
%
% \subsection{Options}\label{sec:options}
%
% The package knows the following options (as key=value syntax):
% \begin{options}
%   \item[col\textit{umns}=\meta{tl}]
%     sets the default column specifier used. Package default is |c|. Both |col|
%     and |columns| will work.
%   \item[align=\meta{tl}]
%     sets the default vertical alignment. Package default is |c|.
%   \item[pad=\meta{choice}]
%     is a choice key. Known choices are:
%     \begin{options}
%       \item[remove]
%         surrounding space is removed using |\hspace*{-\labelsep}|. This is the
%         package default.
%       \item[leave]
%         surrounding space is not removed.
%     \end{options}
%     The optional |*| for \cs{onecell} always sets the opposite of the defined
%     |pad| key.
%   \item[repeat=\meta{bool}]
%     is a boolean key. If it is set to true, the specified \meta{columns} to
%     \cs{onecell} are repeated using the \cs{onecellCycle} column specifier.
%     The optional |!| for \cs{onecell} always sets the opposite of the defined
%     |repeat|. Package default is |true|.
%   \item[default]
%     reverts every other key to the package default.
% \end{options}
%
% \end{documentation}
%
% \begin{implementation}
%
% \section{Implementation}
%
%    \begin{macrocode}
%<*pkg>
%    \end{macrocode}
%
%    \begin{macrocode}
%<@@=onecell>
%    \end{macrocode}
%
%    \begin{macrocode}
\ExplSyntaxOn
%    \end{macrocode}
%
% \subsection{Variables}
%
% \begin{variable}{\l_onecell_align_tl}
%   Stores the default vertical alignment.
%    \begin{macrocode}
\tl_new:N \l_onecell_align_tl
%    \end{macrocode}
% \end{variable}
%
% \begin{variable}{\l_onecell_columns_tl}
%   Stores the default column specifiers (without \cs{onecellCycle} and the
%   repetition).
%    \begin{macrocode}
\tl_new:N \l_onecell_columns_tl
%    \end{macrocode}
% \end{variable}
%
% \begin{variable}{\g_@@_column_already_defined_bool}
%   Set to true if \cs{onecellCycle} was already defined.
%    \begin{macrocode}
\bool_new:N \g_@@_column_already_defined_bool
%    \end{macrocode}
% \end{variable}
%
% \subsection{Messages}
%
%    \begin{macrocode}
\msg_new:nnn { onecell } { unknown_key }
  { The~key~#1~is~unknown~and~therefore~ignored! }
%    \end{macrocode}
%
%    \begin{macrocode}
\msg_new:nnnn { onecell } { column_already_defined }
  { The~column~specifier~'\string\onecellCycle'~is~already~defined. }
  {
    The~package~will~set~the~key~'repeat'~and~its~package~default~to~false.\\
    See~the~documentation~for~what~that~implies.
  }
%    \end{macrocode}
%
% \subsection{Column specifier}
%
% \begin{macro}{\onecellCycle}
% Code from \url{https://tex.stackexchange.com/a/358114/117050}
%    \begin{macrocode}
\ifcsname NC@find@\string\onecellCycle\endcsname
  \msg_error:nn { onecell } { column_already_defined }
  \bool_gset_true:N \g_@@_column_already_defined_bool
\else
  \newcolumntype{\onecellCycle}{!{\d@llarend&\span\@gobbletwo}}
\fi
%    \end{macrocode}
% \end{macro}
%
% \subsection{Options and configuration}
%
%    \begin{macrocode}
\keys_define:nn { onecell }
  {
    ,col     .tl_set_x:N       = \l_onecell_columns_tl
    ,col     .value_required:n = true
    ,columns .meta:n           = { col = #1 }
    ,align   .tl_set_x:N       = \l_onecell_align_tl
    ,pad     .choice:
    ,pad     / remove .code:n  =
      \cs_set_eq:NN \@@_starred:Nn \IfBooleanF
    ,pad     / leave  .code:n  =
      \cs_set_eq:NN \@@_starred:Nn \IfBooleanT
    ,pad     .value_required:n = true
    ,repeat  .choice:
    ,repeat  / true   .code:n  = 
      \cs_set_protected:Nn \@@_processor_excl:n
        { \def\ProcessedArgument{##1} }
    ,repeat  / false  .code:n  = 
      \cs_set_eq:NN \@@_processor_excl:n \ReverseBoolean
    ,repeat  .default:n        = true
    ,unknown .code:n     = {
      \msg_error:nnx { onecell } { unknown_key } { \l_keys_key_tl }
    }
  }
%    \end{macrocode}
%
%    \begin{macrocode}
\bool_if:NTF \g_@@_column_already_defined_bool
  {
    \keys_define:nn { onecell } {
      ,default .meta:n = {
        ,col = c
        ,align = c
        ,pad = remove
        ,repeat = false
      }
      ,default .value_forbidden:n = true
    }
  }
  {
    \keys_define:nn { onecell } {
      ,default .meta:n = {
        ,col = c
        ,align = c
        ,pad = remove
        ,repeat = true
      }
      ,default .value_forbidden:n = true
    }
  }
%    \end{macrocode}
%
%    \begin{macrocode}
\keys_set:nn { onecell } { default }
%    \end{macrocode}
%
%    \begin{macrocode}
\ProcessKeysOptions { onecell }
%    \end{macrocode}
%
% \subsection{Macros}
%
% \subsubsection{User level}
%
% \begin{macro}{\onecell}
%    \begin{macrocode}
\NewDocumentCommand{\onecell}
  {
    s 
    >{ \@@_processor_excl:n }t!
    >{ \@@_processor_Optional_O:Vn \l_onecell_columns_tl }O{}
    O{\l_onecell_align_tl}
    +m
  }{{
    \@@_starred:Nn{#1}{\hspace*{-\tabcolsep}}
    \cs_set_protected_nopar:Nn \@@_tabular: { \begin{tabular}[#4] }
    \IfBooleanTF { #2 }
      { \@@_tabular: { #3 } }
      { \@@_tabular: { #3 \onecellCycle #3 } }
      #5%
    \end{tabular}
    \@@_starred:Nn{#1}{\hspace*{-\tabcolsep}}
  }}
%    \end{macrocode}
% \end{macro}
%
% \begin{macro}{\onecellOptions}
%    \begin{macrocode}
\NewDocumentCommand{\onecellOptions}{ +m }
  { \keys_set:nn { onecell } { #1 } }
%    \end{macrocode}
% \end{macro}
%
% \subsubsection{Auxiliary}
%
% \begin{macro}{\@@_processor_Optional_O:nn, \@@_processor_Optional_O:Vn}
%    \begin{macrocode}
\cs_new_protected:Nn \@@_processor_Optional_O:nn
  {
    \tl_if_empty:nTF { #2 }
      { \def\ProcessedArgument{#1} }
      { \def\ProcessedArgument{#2} }
  }
\cs_generate_variant:Nn \@@_processor_Optional_O:nn { Vn }
%    \end{macrocode}
% \end{macro}
%
% \end{implementation}
%
% \PrintIndex
%
%    \begin{macrocode}
\ExplSyntaxOff
%    \end{macrocode}
%
%    \begin{macrocode}
%</pkg>
%    \end{macrocode}
